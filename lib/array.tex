% commands to emulate arrays in latex

\newcommand\arrCnt[1]{cnt#1}
\newcommand\arrCntVal[1]{\the\value{\arrCnt{#1}}}
\newcommand\arrEnd[1]{end#1}
\newcommand\arrEndVal[1]{\the\value{\arrEnd{#1}}}
\newcommand\arrCreate[1]{
    \newcounter{\arrEnd{#1}}
    \newcounter{\arrCnt{#1}}
}
\newcommand\arrClear[1]{
    \setcounter{\arrEnd{#1}}{0}
}
\newcommand\arrPushVal[2]{
    \arrSetVal{#1}{\arrEndVal{#1}}{#2}
    \addtocounter{\arrEnd{#1}}{1}
}
\newcommand\arrPopVal[1]{
    \addtocounter{\arrEnd{#1}}{-1}
    \arrGetVal{#1}{\arrEndVal{#1}}
}
\newcommand\arrSetVal[3]{
    \expandafter\providecommand\csname #1#2 \endcsname{} % hot fix to avoid error on next line as the command may not have been defined yet
    \expandafter\renewcommand\csname #1#2 \endcsname{#3}
}
\newcommand\arrGetVal[2]{
    \csname #1#2 \endcsname
}
\newcommand\arrPrint[1]{
    \setcounter{\arrCnt{#1}}{0}
    \begin{itemize}
    \loop
    \ifnum \arrCntVal{#1}<\arrEndVal{stack}
        \item \arrGetVal{#1}{\arrCntVal{#1}}
        \stepcounter{\arrCnt{#1}}
        \repeat
    \end{itemize}
}

\newcommand\arrFill[3]{
    \setcounter{\arrEnd{#1}}{#2}
    \setcounter{\arrCnt{#1}}{0}
    \loop
    \ifnum \arrCntVal{#1}<\arrEndVal{#1}
        \arrSetVal{#1}{\arrCntVal{#1}}{#3}
        \stepcounter{\arrCnt{#1}}
        \repeat
}

